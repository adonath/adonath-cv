%% start of file `template.tex'.
%% Copyright 2006-2015 Xavier Danaux (xdanaux@gmail.com), 2020-2021 moderncv maintainers (github.com/moderncv).
%
% This work may be distributed and/or modified under the
% conditions of the LaTeX Project Public License version 1.3c,
% available at http://www.latex-project.org/lppl/.


\documentclass[11pt,a4paper,sans]{moderncv}        % possible options include font size ('10pt', '11pt' and '12pt'), paper size ('a4paper', 'letterpaper', 'a5paper', 'legalpaper', 'executivepaper' and 'landscape') and font family ('sans' and 'roman')

% moderncv themes
\moderncvstyle{classic}                             % style options are 'casual' (default), 'classic', 'banking', 'oldstyle' and 'fancy'
\moderncvcolor{blue}                               % color options 'black', 'blue' (default), 'burgundy', 'green', 'grey', 'orange', 'purple' and 'red'
%\renewcommand{\familydefault}{\sfdefault}         % to set the default font; use '\sfdefault' for the default sans serif font, '\rmdefault' for the default roman one, or any tex font name
%\nopagenumbers{}                                  % uncomment to suppress automatic page numbering for CVs longer than one page

% character encoding
%\usepackage[utf8]{inputenc}                       % if you are not using xelatex ou lualatex, replace by the encoding you are using
%\usepackage{CJKutf8}                              % if you need to use CJK to typeset your resume in Chinese, Japanese or Korean

% adjust the page margins
\usepackage[scale=0.75]{geometry}
\usepackage{multicol}

%\setlength{\hintscolumnwidth}{3cm}                % if you want to change the width of the column with the dates
%\setlength{\makecvheadnamewidth}{10cm}            % for the 'classic' style, if you want to force the width allocated to your name and avoid line breaks. be careful though, the length is normally calculated to avoid any overlap with your personal info; use this at your own typographical risks...


% personal data
\name{Axel}{Donath}
\title{Curriculum Vitae}                               % optional, remove / comment the line if not wanted
%\address{street and number}{postcode city}{country}% optional, remove / comment the line if not wanted; the "postcode city" and "country" arguments can be omitted or provided empty
%\phone[mobile]{+1~(234)~567~890}                   % optional, remove / comment the line if not wanted; the optional "type" of the phone can be "mobile" (default), "fixed" or "fax"
%\phone[fixed]{+2~(345)~678~901}
%\phone[fax]{+3~(456)~789~012}
\email{axeldonath@gmail.com}                               % optional, remove / comment the line if not wanted
%\homepage{www.johndoe.com}                         % optional, remove / comment the line if not wanted

% Social icons
%\social[linkedin]{john.doe}                        % optional, remove / comment the line if not wanted
%\social[xing]{john\_doe}                           % optional, remove / comment the line if not wanted
%\social[twitter]{jdoe}                             % optional, remove / comment the line if not wanted
\social[github]{adonath}                              % optional, remove / comment the line if not wanted
%\social[gitlab]{jdoe}                              % optional, remove / comment the line if not wanted
%\social[stackoverflow]{0000000/johndoe}            % optional, remove / comment the line if not wanted
%\social[bitbucket]{jdoe}                           % optional, remove / comment the line if not wanted
%\social[skype]{jdoe}                               % optional, remove / comment the line if not wanted
\social[orcid]{0000-0003-4568-7005}                  % optional, remove / comment the line if not wanted
%\social[researchgate]{jdoe}                        % optional, remove / comment the line if not wanted
%\social[researcherid]{jdoe}                        % optional, remove / comment the line if not wanted
%\social[telegram]{jdoe}                            % optional, remove / comment the line if not wanted
%\social[googlescholar]{googlescholarid}                % optional, remove / comment the line if not wanted


%\extrainfo{additional information}                 % optional, remove / comment the line if not wanted
%\quote{Some quote}                                 % optional, remove / comment the line if not wanted

% bibliography adjustments (only useful if you make citations in your resume, or print a list of publications using BibTeX)
%   to show numerical labels in the bibliography (default is to show no labels)
%\makeatletter\renewcommand*{\bibliographyitemlabel}{\@biblabel{\arabic{enumiv}}}\makeatother
\renewcommand*{\bibliographyitemlabel}{[\arabic{enumiv}]}
%   to redefine the bibliography heading string ("Publications")
%\renewcommand{\refname}{Articles}

% bibliography with mutiple entries
%\usepackage{multibib}
%\newcites{book,misc}{{Books},{Others}}
%----------------------------------------------------------------------------------
%            content
%----------------------------------------------------------------------------------
\begin{document}
%\begin{CJK*}{UTF8}{gbsn}                          % to typeset your resume in Chinese using CJK
%-----       resume       ---------------------------------------------------------
\makecvtitle

\centering\textcolor{color1}{\rule{16cm}{1pt}}

\begin{multicols}{2}
\begin{itemize}
\item I am a Postdoc researcher at the Max Planck Institue for Nuclear Physics in Heidelberg, where I research in the field of Gamma-Ray astromomy
\item I am interested in the Galactic gamma-ray source population, especially catalog production
\item I am interested in open source software development for astronomical data analysis, including Gammapy, Astropy.
\item I am interested in open source software development for astronomical data analysis, including Gammapy, Astropy.
\end{itemize}
\end{multicols}

\centering\textcolor{color1}{\rule{16cm}{1pt}}

\section{\textbf{Education}}
\cventry{\textbf{Phd} \\ \textcolor{gray}{2014 - 2018}}{}{Department of Physics and Astronomy}{University of Heidelberg}{Thesis: \textit{The Galactic Gamma-ray source Population between 10~GeV and 50~TeV}, advised by W.Hofmann}{}  % Arguments not required can be left empty
\cventry{\textbf{M.Sc.} \\ \textcolor{gray}{2011 - 2014}}{}{Department of Physics and Astronomy}{University of Heidelberg}{Thesis: \textit{Towards the H.E.S.S. Galactic Plane Survey Catalog}, advised by W.Hofmann}{}  % Arguments not required can be left empty
\cventry{\textbf{B.Sc.} \\ \textcolor{gray}{2007 - 2011}}{}{Department of Physics and Astronomy}{University of Heidelberg}{Specialized in Astronomy and Image Processing}{}



\section{\textbf{Experience}}

\subsection{Employment}
%------------------------------------------------
\cvitem{\textbf{MPIK} \\ \textcolor{gray}{2018--present}}{Postdoctoral researcher, non-thermal astrophysics, Max-Planck-Institut for Nuclear Physics, Heidelberg. Supervised by Jim Hinton}
\cvitem{\textbf{MPIK} \\ \textcolor{gray}{2014--2018}}{PhD Student, high energy astrophysics, Max-Planck-Institut for Nuclear Physics, Heidelberg. Supervised by Werner Hofmann}
\cvitem{\textbf{HCI} \\ \textcolor{gray}{2012--2014}}{Student assistent, Heidelberg Collaboratory for Image Processing (HCI), Heidelberg. Supervised by Bernd Jähne.}
\cvitem{\textbf{Halle02} \\ \textcolor{gray}{2006--2007}}{Voluntary year in culture at \textit{Atelier Kontrast}, Heidelberg}
%------------------------------------------------

\subsection{Formal Teaching}
\cvitem{Spring 2016}{Tutor for \textit{Python programming for scientists}, University of Heidelberg}
\cvitem{Fall 2015}{Tutor for \textit{Python programming for scientists}, University of Heidelberg}
\cvitem{Autumn 2009}{\textit{Basiskurs Schl\"usselkompetenzen}, University of  Heidelberg}

\subsection{Service}
\cvitem{\textbf{PyGamma19} \\ \textcolor{gray}{2019}}{Member of the SoC and LoC for the 1st Python in Gamma-Ray Astronomy Conference}
\cvitem{\textbf{PyGamma15} \\ \textcolor{gray}{2015}}{Member of the SoC and LoC for the 2nd Python in Gamma-Ray Astronomy Conference}
\cvitem{\textbf{Gammapy} \\ \textcolor{gray}{2015-present}}{Organisation of mutiple Gammapy coding sprints (~3 per year) and Gammapy developer calls}

\subsection{Mentoring}
\cvitem{\textbf{GSoC} \\ \textcolor{gray}{Summer 2015}}{Mentoring of Google Summer of Code student Patty Carroll for Astropy}


%----------------------------------------------------------------------------------------
%	AWARDS SECTION
%----------------------------------------------------------------------------------------

\section{\textbf{Awards}}
\cvitem{\textbf{ICVS} \\ \textcolor{gray}{June 2013}}{Recipient of the \textit{Best Paper Award} at the International Conference on Computer Vision Systems for the contribution \textit{Is Crowdsourcing feasible for optical flow ground truth estimation?
}
                   }

%----------------------------------------------------------------------------------------
%	COMPUTER SKILLS SECTION
%----------------------------------------------------------------------------------------

\section{\textbf{Computing}}
\subsection{Skills}
\cvitem{Expert}{\textsc{Python},  \textsc{Numpy}, \textsc{Scipy}, \textsc{Matplotlib}, \textsc{Scikit-Learn}, \textsc{PyQt}, \textsc{Astropy}}
\cvitem{Intermediate}{\textsc{C++}, \LaTeX, \textsc{Linux}}
\cvitem{General}{\textsc{Git}, \textsc{Pytest}, \textsc{Sphinx}}

\subsection{Software}
\cvitem{2013}{Google Summer of Code project for Astropy}
\cventry{2012--Present}{Gammapy}{I am founder and co-developer of the Gammapy Python package for Gamma-Ray astronomy}{}{}{}
\cventry{2014--Present}{Astropy}{I am sub-package maintainer of the Astropy convolution sub package}{}{}{}

\section{\textbf{Talks}}
\subsection{Astronomy}
\cvitem{\textbf{Gamma} \\ \textcolor{gray}{June 2016}}{The H.E.S.S. Galactic Plane Survey}
\cvitem{\textbf{TeVPa} \\ \textcolor{gray}{June 2016}}{The H.E.S.S. Galactic Plane Survey}
\cvitem{\textbf{H.E.S.S.}\\ \textcolor{gray}{2014-2018}}{Various presentations on the H.E.S.S. Galactic Plane Survey for the H.E.S.S. collaboration}


** Invited tutorials
\subsection{Software}
\cvitem{\textbf{Herbsttagung DAG} \\ \textcolor{gray}{Autumn 2014}}{Astropy - A community developed core package for Astronomy in Python}
\cvitem{\textbf{PyGamma15} \\ \textcolor{gray}{November 2015}}{An overview of Astropy, Sherpa and Gammapy}
\cvitem{\textbf{PyGamma19} \\ \textcolor{gray}{March 2019}}{Gammapy - A Python Package for Gamma-Ray Astronomy}
\cvitem{\textbf{**Asterics-Obelics} \\ \textcolor{gray}{2017, 2018, 2019}}{\textit{Astropy} tutorial, 1st ASTERICS-OBELICS International School}
\cvitem{\textbf{**Sexten} \\ \textcolor{gray}{June 2019}}{\textit{Gammapy} tutorial, Multimessenger Data Analysis in the CTA era Summer School, Sexten}

\cvitem{\textbf{CTA / H.E.S.S.} \\ \textcolor{gray}{2014 -- present}}{Various Gammapy talks and tutorials at
H.E.S.S and CTA collaboration meetings}


%\section{Skill matrix}
%\cvitem{Skill matrix}{Alternatively, provide a skill matrix to show off your skills}
%%% Skill matrix as an alternative to rate one's skills, computer or other.
%
%%% Adjusts width of skill matrix columns.
%%% Usage \setcvskillcolumns[<width>][<factor>][<exp_width>]
%%% <width>, <exp_width> should be lengths smaller than \textwidth, <factor> needs to be between 0 and 1.
%%% Examples:
%% \setcvskillcolumns[5em][][]%    adjust first column. Same as \setcvskillcolumns[5em]
%% \setcvskillcolumns[][0.45][]%   adjust third (skill) column. Same as \setcvskillcolumns[][0.45]
%% \setcvskillcolumns[][][\widthof{``Year''}]%     adjust fourth (years) column.
%% \setcvskillcolumns[][0.45][\widthof{``Year''}]%
%% \setcvskillcolumns[\widthof{``Languag''}][0.48][]
%% \setcvskillcolumns[\widthof{``Languag''}]%
%
%%% Adjusts width of legend columns. Usage \setcvskilllegendcolumns[<width>][<factor>]
%%% <factor> needs to be between 0 and 1. <width> should be a length smaller than \textwidth
%%% Examples:
%% \setcvskilllegendcolumns[][0.45]
%% \setcvskilllegendcolumns[\widthof{``Legend''}][0.45]
%% \setcvskilllegendcolumns[0ex][0.46]% this is usefull for the banking style
%
%%% Add a legend if you are using \cvskill{<1-5>} command or \cvskillentry
%%% Usage \cvskilllegend[*][<post_padding>][<first_level>][<second_level>][<third_level>][<fourth_level>][<fifth_level>]{<name>}
%% \cvskilllegend % insert default legend without lines
%\cvskilllegend*[1em]{}% adjust post spacing
%% \cvskilllegend*{Legend}%  Alternatively add a description string
%%% adjust the legend entries for other languages, here German
%% \cvskilllegend[0.2em][Grundkenntnisse][Grundkenntnisse und eigene Erfahrung in Projekten][Umfangreiche Erfahrung in Projekten][Vertiefte Expertenkenntnisse][Experte\,/\,Spezialist]{Legende}
%
%%% Alternative legend style with the first three skill levels in one column
%%% Usage \cvskillplainlegend[*][<post_padding>][<first_level>][<second_level>][<third_level>][<fourth_level>][<fifth_level>]{<name>}
%% \setcvskilllegendcolumns[][0.6]%  works for classic, casual, banking
%% \setcvskilllegendcolumns[][0.55]%  works better for oldstyle and fancy
%% \cvskillplainlegend{}
%% \cvskillplainlegend[0.2em][Grundkenntnisse][Grundkenntnisse und eigene Erfahrung in Projekten][Umfangreiche Erfahrung in Projekten][Vertiefte Expertenkenntnisse][Experte/Guru]{Legende}
%
%%% Add a head of the skill matrix table with descriptions.
%%% Usage \cvskillhead[<post_padding>][<Level>][<Skill>][<Years>][<Comment>]%
%\cvskillhead[-0.1em]%   this inserts the standard legend in english and adjust padding
%%% Adjust head of the skill matrix for other languages
%% \cvskillhead[0.25em][Level][F\"ahigkeit][Jahre][Bemerkung]
%
%%% \cvskillentry[*][<post_padding>]{<skill_cathegory>}{<0-5>}{<skill_name>}{<years_of_experience>}{<comment>}%
%%% Example usages:
%\cvskillentry*{Language:}{3}{Python}{2}{I'm so experienced in Python and have realised a million projects. At least.}
%\cvskillentry{}{2}{Lilypond}{14}{So much sheet music! Man, I'm the best!}
%\cvskillentry{}{3}{\LaTeX}{14}{Clearly I rock at \LaTeX}
%\cvskillentry*{OS:}{3}{Linux}{2}{I only use Archlinux btw}% notice the use of the starred command and the optional
%\cvskillentry*[1em]{Methods}{4}{SCRUM}{8}{SCRUM master for 5 years}
%%% \cvskill{<0-5>} command
%% \cvitem{\textbackslash{cvskill}:}{Skills can be visually expressed by the \textbackslash{cvskill} command, e.g. \cvskill{2}}

% Publications from a BibTeX file without multibib
%  for numerical labels: \renewcommand{\bibliographyitemlabel}{\@biblabel{\arabic{enumiv}}}% CONSIDER MERGING WITH PREAMBLE PART
%  to redefine the heading string ("Publications"): \renewcommand{\refname}{Articles}
\nocite{*}
\bibliographystyle{plain}
\bibliography{adonath-cv}                        % 'publications' is the name of a BibTeX file

% Publications from a BibTeX file using the multibib package
%\section{Publications}
%\nocitebook{book1,book2}
%\bibliographystylebook{plain}
%\bibliographybook{publications}                   % 'publications' is the name of a BibTeX file
%\nocitemisc{misc1,misc2,misc3}
%\bibliographystylemisc{plain}
%\bibliographymisc{publications}                   % 'publications' is the name of a BibTeX file

%\clearpage\end{CJK*}                              % if you are typesetting your resume in Chinese using CJK; the \clearpage is required for fancyhdr to work correctly with CJK, though it kills the page numbering by making \lastpage undefined
\end{document}


%% end of file `template.tex'.

